%%%%%%%%%%%%%%%%%%%%%%%%%%%%%%%%%%%%%%%%%%%%%%%%%%%%%%%%%%%%%%%%%%%%%%%%%%%%%
%
% Vorlage für Seminararbeiten im Institut für Verteilte Systeme
% 
% HINWEISE
% 
%  1. Bei Nutzung für Seminarausarbeitungen darf insbesondere die Schriftart
%     und -größe nicht angepasst werden.
%  2. Die Vorlage unterstützt deutsche und englische Ausarbeitungen durch
%     Anpassung der babel Paketoptionen.
%  3. Folgende Angaben sollen angepasst werden:
%     - Titel der Arbeit
%     - Name und E-Mail-Adresse des Autors
%     - Titel des Seminars
%     - Semester
%  4. Die Vorlage sieht eine Lizensierung unter CC-BY-SA vor, die jedoch
%     nicht verpflichtend ist. Falls nicht gewünscht, bitte alle \thanks
%     Befehle auskommentieren.
%     Die gewählte Lizenz (CC-BY-SA) ist kompatibel mit einer möglichen
%     Veröffentlichung auf dem Volltextserver der Uni Ulm
%     (http://vts.uni-ulm.de).
%
%%%%%%%%%%%%%%%%%%%%%%%%%%%%%%%%%%%%%%%%%%%%%%%%%%%%%%%%%%%%%%%%%%%%%%%%%%%%%

% Based on the IEEE Journal style.
\documentclass[10pt,a4paper,compsoc,peer review papers]{IEEEtran}

\usepackage{graphicx}
\usepackage[cmex10]{amsmath}
\usepackage[ngerman]{babel} % Deutsche Ausarbeitung
% \usepackage[USenglish]{babel} % Englische Ausarbeitung
\usepackage{url}
\usepackage[colorlinks]{hyperref}
\usepackage{color}  % Definition von Linkfarben
\definecolor{DarkRed}{rgb}{0.5,0,0}
\hypersetup{
  colorlinks,
  citecolor=DarkRed,
  linkcolor=DarkRed,
  urlcolor=blue}
\usepackage[utf8]{inputenc}
\usepackage[T1]{fontenc}  % Korrekte Umlaute im Output

\newcommand\IEEEfirstsection[1]{%
  \noindent\raisebox{2\baselineskip}[0pt][0pt]%
  {\parbox{\columnwidth}{#1%
  \global\everypar=\everypar}}%
  \vspace{-1\baselineskip}\vspace{-\parskip}\par
}

\newcommand\cclicense{{\normalfont\sffamily\bfseries CC-BY-SA}}
\IfFileExists{ccicons.sty}{%
\usepackage{ccicons}
\renewcommand\cclicense{\ccbysa}
}

\begin{document}

\title{LEX 2016: Bestimmung der Wolkenhöhe mittels Pyrgeometer}

\author{%
\IEEEauthorblockN{Lukas Kluft}\\
\IEEEauthorblockA{\url{lukas.kluft@gmail.com}}%
%
%\iflanguage{ngerman}{%
%\thanks{%
%\cclicense{}
%Diese Arbeit steht unter einer
%Creative Commons Namensnennung - Weitergabe unter gleichen Bedingungen
%3.0 Deutschland Lizenz.}%
%\thanks{\url{http://creativecommons.org/licenses/by-sa/3.0/de/}}%
%}{ % Englische Ausarbeitung
%\thanks{%
%\cclicense{}
%This work is licensed under a
%Creative Commons Attribution-ShareAlike 4.0 International License.}%
%\thanks{\url{http://creativecommons.org/licenses/by-sa/4.0/}}%
%}%
}

\IEEEpubid{\sffamily%
\makebox[\columnwidth]{\hfill Lehrexkursion Fehmarn}%
\makebox[\columnsep]{$\cdot$}%
\makebox[\columnwidth]{SS 2016,
Meteorologisches Institut - CEN, Universit"at Hamburg\hfill}}

\IEEEtitleabstractindextext{%
\begin{abstract}
foo bar
\end{abstract}%
}

% make the title area
\maketitle

\IEEEfirstsection{%
    \section{Hintergrund}\label{sec:hintergrund}}
Pyrgeometer messen die aus dem Halbraum eintreffende atmosphärische
Gegenstrahlung (5-50\,$\mu$m).  Die Stärke der Gegenstrahlung hängt vom
Zustand der Atmosphäre ab; bei Bewölkung ist diese deutlich stärker als bei
wolkenfreien Verhältnissen.  Zusätzlich hängt die Emission von der Temperatur
ab; warme Körper strahlen stärker als kalte.  Diese beiden Effekte ermöglichen 
es über die atmosphärische Gegenstrahlung Rückschlüsse auf die Temperatur der
Wolkenunterkante zu ziehen.  Mit Hilfe zusätzlicher Annahmen über das
Temperaturprofil kann so die Höhe der Wolke abgeschätzt werden.

\section{Konzept}\label{sec:konzept}
\begin{enumerate}
  \item Berechnung der Wolkentemperatur aus den Strahlungsmessungen des Pyrgeometers.\\
  Die gemessene Leistung der atmosphärischen Gegenstrahlung lässt sich mit Hilfe des
  Stefan-Boltzmann-Gesetzes in eine Temperatur umrechnen.  
  \[ E = \sigma T^4 \]

  \item Zuordnung der Wolkentemperatur zu einer Höhe.\\
  Um aus der Temperatur der Wolke eine Abschätzung ihrer Höhe zu gewinnen,
  müssen Informationen über das Temperaturprofil der Atmosphäre vorliegen.
  Hierzu können Ansätze unterschiedlicher Komplexität gewählt werden:
  \begin{itemize}
    \item adiabatische Abnahme der Temperatur ausgehende von der Bodentemperatur
    \item Standardatmosphäre mit angepasster Bodentemperatur
    \item Radiosondenaufstiege
  \end{itemize}
\end{enumerate}

\section{Mögliche Probleme}\label{sec:probleme}
Zusätzlich zu den genannten Herausforderungen können weitere Probleme den
Erfolg des Experimentes beeinflussen:
\begin{itemize}
  \item Abhängigkeit der atmosphärischen Gegenstrahlung vom Bedeckungsgrad.
  \item Abweichungen der Messungen durch variable Wasserdampfabsorption.
\end{itemize}

\end{document}
