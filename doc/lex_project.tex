\documentclass[a4paper,fleqn]{article}
%% Seitenaufbau
\usepackage[top=3cm, bottom=2.5cm, left=3.5cm, right=3.5cm]{geometry}

%% Schriftbild
\usepackage{lmodern}  % Latin Modern Zeichensatz
\usepackage[utf8]{inputenc}  % Unterstuetzung von Umlauten im Quelltext
\usepackage[T1]{fontenc}  % Korrekte Umlaute im Output
\usepackage[ngerman]{babel}  % Silbentrennung nach neuer Rechtschreibung
\renewcommand{\familydefault}{\sfdefault}  % Serifenlose Schrift
\usepackage{setspace}\onehalfspacing  % 1.5-facher Zeilenabstand
\renewcommand{\arraystretch}{1.5}  % 1.5-facher Zeilenabstand (Tabellen)
\setlength{\parindent}{0pt}  % Keine Einrueckung am Beginng von Absaetzen
\usepackage{fancyhdr}
\pagestyle{fancy}
\renewcommand{\headrulewidth}{0.5pt}
\lhead{\LaTeX-Vorlage}
\rhead{Lukas Kluft}
\chead{}
\cfoot{\thepage}  % Richtige Schriftart fuer Seitenzahlen
\sloppy  % Weniger strikte Silbentrennung

%% Literaturverzeichnis (BibTex)
\usepackage{natbib}
\bibliographystyle{apalike}  % Layout des Literaturverzeichnisses

%% Verlinkung von Inhaltsverzeichnis, Bildern und Formeln
\usepackage[pagebackref]{hyperref}  % Verlinkung von URLs und Referenzen
\usepackage{color}  % Definition von Linkfarben
\definecolor{DarkRed}{rgb}{0.5,0,0}
\hypersetup{
  colorlinks,
  citecolor=DarkRed,
  linkcolor=DarkRed,
  urlcolor=blue}

%% Mathematikumgebung
\usepackage{mathtools}
\usepackage{amssymb}  % Erweiterte Bibliothek mathematischer Symbole
\usepackage{euler}  % Serifenlose Schrift in Formelumgebungen
\usepackage[makeroom]{cancel}  % Durchstreichen von Termen
\renewcommand{\deg}{\ensuremath{^{\circ}}}  % Grad Zeichen im Text
\renewcommand{\epsilon}{\varepsilon}  % Nutze "richtiges" Epsilon

%% Grafikumgebungen
\usepackage{graphicx}  % Erweiterte Grafikumgebung
\usepackage{float}  % Automatische Positionierung von Bildern
\usepackage{floatflt}  % Grafiken im Text einbetten
\usepackage{subcaption}  %  Bildunterschriften fuer subfigures

%% Formeln, Bilder und Tabellen pro section nummerieren
\numberwithin{equation}{section}
\numberwithin{figure}{section}
\numberwithin{table}{section}

%% Listenumgebung
\usepackage{enumerate}
\usepackage{textcomp}  % Korrekte serifenlose Aufzählungszeichen

%% Farbige Umrahmungen
\usepackage{framed}
\definecolor{shadecolor}{rgb}{0.9,0.9,0.9}

%% Blindtext zum Testen des Layouts
\usepackage{blindtext}

%% Inhalt der Titelseite
\title{Bestimmung der Wolkenhöhe mittels Pyrgeometer}
\author{Seminar zur Lehrexkursion}
\date{\today}

%%%%%%%%%%%%%%%%%%%%%%%%%%%%%%%%%%%%%%%%%%%%%%%%%%%%%%%%%%%%%%%%%%%%%%
\begin{document}
\maketitle
%\thispagestyle{empty}\pagestyle{empty}
%\tableofcontents
%\newpage\pagestyle{fancy}

\section{Hintergrund}\label{sec:hintergrund}
Pyrgeometer messen die aus dem Halbraum eintreffende atmosphärische
Gegenstrahlung (5-50\,$\mu$m).  Die Stärke der Gegenstrahlung hängt vom
Zustand der Atmosphäre ab; bei Bewölkung ist diese deutlich stärker als bei
wolkenfreien Verhältnissen.  Zusätzlich hängt die Emission von der Temperatur
ab; warme Körper strahlen stärker als kalte.  Diese beiden Effekte ermöglichen 
es über die atmosphärische Gegenstrahlung Rückschlüsse auf die Temperatur der
Wolkenunterkante zu ziehen.  Mit Hilfe zusätzlicher Annahmen über das
Temperaturprofil kann so die Höhe der Wolke abgeschätzt werden.

\section{Konzept}\label{sec:konzept}
\begin{enumerate}
  \item Berechnung der Wolkentemperatur aus den Strahlungsmessungen des Pyrgeometers.\\
  Die gemessene Leistung der atmosphärischen Gegenstrahlung lässt sich mit Hilfe des
  Stefan-Boltzmann-Gesetzes in eine Temperatur umrechnen.  
  \[ E = \sigma T^4 \]
  
  \item Zuordnung der Wolkentemperatur zu einer Höhe.\\
  Um aus der Temperatur der Wolke eine Abschätzung ihrer Höhe zu gewinnen,
  müssen Informationen über das Temperaturprofil der Atmosphäre vorliegen.
  Hierzu können Ansätze unterschiedlicher Komplexität gewählt werden:
  \begin{itemize}
    \item adiabatische Abnahme der Temperatur ausgehende von der Bodentemperatur
    \item Standardatmosphäre mit angepasster Bodentemperatur
    \item Radiosondenaufstiege
  \end{itemize}
\end{enumerate}

\section{Mögliche Probleme}\label{sec:probleme}
Zusätzlich zu den genannten Herausforderungen können weitere Probleme den
Erfolg des Experimentes beeinflussen:
\begin{itemize}
  \item Abhängigkeit der atmosphärischen Gegenstrahlung vom Bedeckungsgrad.
  \item Abweichungen der Messungen durch variable Wasserdampfabsorption.
\end{itemize}

% BibTex-Literaturverzeichnis (Pfad ohne Dateiendung angeben)
%\bibliography{../Literaturverzeichnis}
\end{document}

