%%%%%%%%%%%%%%%%%%%%%%%%%%%%%%%%%%%%%%%%%%%%%%%%%%%%%%%%%%%%%%%%%%%%%%%%%%%%%
%
% Vorlage für Seminararbeiten im Institut für Verteilte Systeme
% 
% HINWEISE
% 
%  1. Bei Nutzung für Seminarausarbeitungen darf insbesondere die Schriftart
%     und -größe nicht angepasst werden.
%  2. Die Vorlage unterstützt deutsche und englische Ausarbeitungen durch
%     Anpassung der babel Paketoptionen.
%  3. Folgende Angaben sollen angepasst werden:
%     - Titel der Arbeit
%     - Name und E-Mail-Adresse des Autors
%     - Titel des Seminars
%     - Semester
%  4. Die Vorlage sieht eine Lizensierung unter CC-BY-SA vor, die jedoch
%     nicht verpflichtend ist. Falls nicht gewünscht, bitte alle \thanks
%     Befehle auskommentieren.
%     Die gewählte Lizenz (CC-BY-SA) ist kompatibel mit einer möglichen
%     Veröffentlichung auf dem Volltextserver der Uni Ulm
%     (http://vts.uni-ulm.de).
%
%%%%%%%%%%%%%%%%%%%%%%%%%%%%%%%%%%%%%%%%%%%%%%%%%%%%%%%%%%%%%%%%%%%%%%%%%%%%%

% Based on the IEEE Journal style.
\documentclass[10pt,a4paper,compsoc,peer review papers]{IEEEtran}

\usepackage{graphicx}
\usepackage[cmex10]{amsmath}
\usepackage[ngerman]{babel} % Deutsche Ausarbeitung
% \usepackage[USenglish]{babel} % Englische Ausarbeitung
\usepackage{url}
\usepackage[colorlinks]{hyperref}
\usepackage{color}  % Definition von Linkfarben
\definecolor{DarkRed}{rgb}{0.5,0,0}
\hypersetup{
  colorlinks,
  citecolor=DarkRed,
  linkcolor=DarkRed,
  urlcolor=blue}
\usepackage[utf8]{inputenc}
\usepackage[T1]{fontenc}  % Korrekte Umlaute im Output

\newcommand\IEEEfirstsection[1]{%
  \noindent\raisebox{2\baselineskip}[0pt][0pt]%
  {\parbox{\columnwidth}{#1%
  \global\everypar=\everypar}}%
  \vspace{-1\baselineskip}\vspace{-\parskip}\par
}

\newcommand\cclicense{{\normalfont\sffamily\bfseries CC-BY-SA}}
\IfFileExists{ccicons.sty}{%
\usepackage{ccicons}
\renewcommand\cclicense{\ccbysa}
}

\begin{document}

\title{LEX 2016: Bestimmung der Wolkenhöhe mittels Pyrgeometer}

\author{%
\IEEEauthorblockN{Lukas Kluft, Timorsha Rafiq-Dost}\\
\IEEEauthorblockA{\url{lukas.kluft@gmail.com}, \url{timorsha@live.de}}%
%
%\iflanguage{ngerman}{%
%\thanks{%
%\cclicense{}
%Diese Arbeit steht unter einer
%Creative Commons Namensnennung - Weitergabe unter gleichen Bedingungen
%3.0 Deutschland Lizenz.}%
%\thanks{\url{http://creativecommons.org/licenses/by-sa/3.0/de/}}%
%}{ % Englische Ausarbeitung
%\thanks{%
%\cclicense{}
%This work is licensed under a
%Creative Commons Attribution-ShareAlike 4.0 International License.}%
%\thanks{\url{http://creativecommons.org/licenses/by-sa/4.0/}}%
%}%
}

\IEEEpubid{\sffamily%
\makebox[\columnwidth]{\hfill Lehrexkursion Fehmarn}%
\makebox[\columnsep]{$\cdot$}%
\makebox[\columnwidth]{SS 2016,
Meteorologisches Institut - CEN, Universit"at Hamburg\hfill}}

\IEEEtitleabstractindextext{%
\begin{abstract}
Pyrgeometer messen die aus dem Halbraum eintreffende atmosphärische
Gegenstrahlung (5-50\,$\mu$m). Die Stärke der Gegenstrahlung hängt vom
Zustand der Atmosphäre ab; bei Bewölkung ist diese deutlich stärker als bei
wolkenfreien Verhältnissen. Zusätzlich hängt die Emission von der Temperatur
ab; warme Körper strahlen stärker als kalte. Diese beiden Effekte ermöglichen 
es über die atmosphärische Gegenstrahlung Rückschlüsse auf die Temperatur der
Wolkenunterkante zu ziehen. Mit Hilfe zusätzlicher Annahmen über das
Temperaturprofil kann so die Höhe der Wolkenunterkante abgeschätzt werden.
\end{abstract}%
}

\maketitle

\IEEEfirstsection{\section{Grundlagen}\label{sec:grundlagen}}
Das Wasser in Wolken absorbiert terrestrische Strahlung sehr stark. Dadurch
verändern Wolken die optische Dicke und damit die Wichtungsfunktion gegenüber
dem wolkenlosen Fall. Die Wichtung von Höhen überhalb der Wolken wird
geschwächt, die der Wolkenschicht erhöht. Durch die generelle Abnahme der
Temperatur mit der Höhe erhöht sich durch das Vorhandensein von Wolken die am
Boden ankommende terrestrische Strahlung. Je tiefer die Wolke, desto stärker
der Anstieg der Strahlung gegenüber dem wolkenlosen Fall. Zusätzlich hängt die
Emission von der Temperatur ab, warme Körper strahlen stärker als kalte. Diese
beiden Effekte ermöglichen es über die atmosphärische Gegenstrahlung
Rückschlüsse auf die Temperatur der Wolkenunterkante zu ziehen. Mithilfe
zusätzlicher Annahmen über das Temperaturprofil kann so die Höhe der
Wolkenunterkante abgeschätzt werden. 

\section{Motivation}
Wolken gehören zu den variabelsten und inhomogensten Bestandteilen der
Atmosphäre. Sie decken einen Größenbereich von einigen Metern, wie z.B
Cumuluswolken, bis zu mehreren hundert Kilometern (Frontensysteme) ab. Zudem
erstreckt sich auch die zeitliche Variabilität über mehrere Größenordnungen.
Von flachen Schönwettercumulus, dessen Lebenszeit mehrere Stunden beträgt, bis
zu mehreren Tagen bestehende Stratuswolken.

Die Existenz, Häufigkeit und die Art von Wolken im Allgemeinen sowie die
Wolkenhöhe im Speziellen spielen eine eminent große Rolle für das Wetter und
Klima. Neben Niederschlagsprozessen beeinflussen sie auch die Strahlungsbilanz
der Erde. Die auf die Erde einfallende Solarstrahlung wird über Wolken und
Atmosphäre teilweise in das Weltall zurückgeworfen, weshalb einige Wolkentypen
zu Abkühlung des Planeten tendieren. Die abgegebene terrestrische Strahlung vom
Erdboden, welche in Richtung Himmel gerichtet ist, wird von Wolken absorbiert.
Die dadurch bereitgestellte Energie kann genutzt werden, um langwellige
Strahlung in Richtung Erdoberfläche zu emittieren. Wolken haben dementsprechend
auch einen erwärmenden Effekt auf die Erde und sind ausschlaggebend für den
Treibhauseffekt [IPCC, 2013]. Ob eine Wolke erwärmend oder abkühlend auf die
Atmosphäre wirkt hängt entscheidend von der Albedo, optischen Dicke und der
thermischen Ausstrahlung, also der Wolkenhöhe ab.

Außerdem hat die Wolkenunterkante auch praktische Bedeutung, wie zum Beispiel
in der Luftfahrtberatung. Bei Sichtflugbedingungen etwa darf eine gewisse
Wolkenuntergrenze nicht unterschritten werden.

Hinsichtlich der Wichtigkeit der Wolkenhöhe wurden im Rahmen der Lehrexkursion
(LEX) auf Fehmarn Strahlungsmessungen mit einem Pyrgeometer durchgeführt.
Pyrgeometer messen dabei die aus dem Halbraum eintreffende atmosphärische
Gegenstrahlung in einem Wellenlängenbereich zwischen 4,5 $\mu$m und 45 $\mu$m.
Die Grundidee und Motivation dieses Versuches liegt darin, das
Leistungspotenzial und die Fähigkeiten eines einfachen Messinstruments wie
einem Pyrgeometer zu testen. Die langwellige Einstrahlung und die Temperatur,
welche mittels Pyrgeometer beziehungsweise einem Thermometer ermittelt werden,
sind jene Messgrößen, die für diesen speziellen Versuch von Bedeutung sind. Es
soll geklärt werden, inwiefern und mit welcher Genauigkeit es möglich ist
atmosphärische Zustandsgrößen aus diesen einfachen Messgrößen abzuleiten.

Im Folgenden werden dafür zunächst der zugrunde liegende Messaufbau in
Abschnitt 2 vorgestellt. Abschnitt 3 erläutert den über die Messperiode
gewonnen Datensatz. Darauf folgend werden in Abschnitt 4 die Ergebnisse des
Projekts präsentiert.  Die Konzentration liegt einerseits in der Ermittlung der
Wolkenbasishöhe mittels der langwelligen Einstrahlung. Da die langwellige
Einstrahlung zudem vom Wasserdampfgehalt der Atmosphäre abhängig ist, können
die wolkenlosen Tage dazu genutzt werden um die Wasserdampfsäule zu bestimmen.
Abschließend folgen in Abschnitt 5 ein Fazit und ein kurzer Ausblick.

\section{Messaufbau}
Für die Messung der Strahlung wurde eine Strahlungsstation (Strahlungsgarten)
auf einer Wiese installiert. Der genauere Messaufbau des Strahlungsgartens kann
der Abbildung entnommen werden (Foto). Auf der Wiese wird die Einstrahlung mit
je zwei nach oben gerichteten Strahlungssensoren und die Ausstrahlung mit je
zwei nach unten gerichteten Strahlungssensoren von EIGENBRODT gemessen. Für die
langwellige Strahlung aus dem oberen und unteren Halbraum werden Pyrgeometer
und für die kurzwellige Strahlung Pyranometer verwendet. Die Instrumente
befinden sich etwa einen Meter über der Erdoberfläche. (Zusätzlich dazu wurde
in unmittelbarer Nähe ein Schattenring aufgestellt, welcher mittels GPS
genordet worden ist.) Die exakten geographischen Positionen des Schattenringes
sowie des Strahlungsgartens kann der Tabelle entnommen werden. 

Zusätzlich wurden ein Ceilometer von VAISALA und das Mikrowellenradiometer
HATPRO als Referenz genutzt, um Ergebnisse mit dem Pyrgeometer vergleichen zu
können. Das Ceilometer und Radiometer sind ebenfalls auf der Wiese in
unmittelbarer Nähe zum Strahlungsgarten betrieben wurden. Das Ceilometer sendet
Laserimpulse aus und detektiert das aus der Atmosphäre zurückgestreute Licht.
Wolken und Staubpartikel streuen das Laserlicht. Aus der Laufzeit der Signale
und Lichtgeschwindigkeit wird die Entfernung zum Ort der Streuung berechnet.
Eine einfache Anwendung ist daher die hochauflösende Entfernungsmessung von
Wolkenunterkanten. Das HATPRO misst die thermische Ausstrahlung der Atmosphäre.
Durch eine entsprechende Auswahl der Empfangsfrequenzen kann die Emission von
Mikrowellenstrahlung bestimmter Spurengase, von Flüssigwasser sowie von
Eiskristallen empfangen werden. So befindet sich beispielsweise bei Frequenzen
von 20 bis 30 GHz ein starkes Absorptionsband von Wasserdampf. Messungen der
Strahlungstemperatur an der Flanke dieses Absorptionsbandes erlaubt
Abschätzungen zum Gesamtwasserdampfgehalt (Integrated Water Vapour IWV). Die
aufgenommenen Wolkenbasishöhen und der IWV vom Ceilometer bzw. HATPRO dienen im
folgenden als Referenzdaten. 

Das Messgelände ist zudem gekennzeichnet durch diverse Besonderheiten: In
südlicher Richtung vom Messaufbau befindet sich ein Funkturm. Außerdem befindet
sich der Aufbau in direkter Küstennähe. Bei tiefstehender Sonne kann es zu
Einfluss durch Reflexionen auf der Wasseroberfläche kommen. Der Einfluss auf
die Messung ist jedoch wahrscheinlich gering, da die verwendeten Instrumente
nah am Horizont nicht sensitiv sind. 

\section{Datensatz}
Die Lehrexkursion fand in dem Zeitraum vom 29. August bis 9. September 2016
statt. Jedoch stehen die Daten nicht für die komplette Periode zur Verfügung.
Der letzte Tag wurden genutzt um die Geräte abzubauen, daher können
ausschließlich die Daten vom 19.08. - 07.09.2016 berücksichtigt werden. Es ist
darauf hinzuweisen, dass ein 15-minütiger Stromausfall am 01.09.2016 von 14.20
UTC bis 14.35 UTC dafür gesorgt hatte, dass auch hier keine Daten verfügbar
sind. 

Die Messungen des Pyrgeometers und Ceilometers sind resistent gegenüber
Wettereinwirkungen wie Regen, sodass permanent verwertbare Daten von der
atmosphärischen Gegenstrahlung bzw. Wolkenunterkantenhöhe geliefert werden (
Mittelung erwähnen) konnten. Beide Geräte liefern Daten in einminütiger
Auflösung. Die Messungen des Radiometers liegen in zehnminütiger Auflösung vor.
Beim HATPRO-gibt es allerdings Einschränkungen, sobald es regnet; während
Regen-Events sind die IWV-Messdaten unbrauchbar. 

\section{Strahlungstransfer}\label{sec:strahlungstransfer}
Pyrgeometer messen den gesamten Strahlungsfluss im langwelligen
Frequenzspektrum zwischen 3\,THz und 60\,THz. Um genauere Informationen darüber
zu gewinnen, aus welchen Teilen der Atmosphäre die gemessene Strahlung stammt,
wurden Strahlungstransferrechnungen durchgeführt.

Für die Berechnung wurde der Atmospheric Radiative Transfer Simulator (ARTS)
verwendet \cite{Eriksson2011}. ARTS ist ein physikalisches
Strahlungstransfermodell für den Millimeter- und Submillimeterbereich des
elektromagnetischen Strahlungsspektrums.

\section{Ergebnisse}\label{sec:ergebnisse}

\section{Schlussfolgerungen}\label{sec:schlussfolgerungen}

% bibliography
\bibliographystyle{IEEEtranS}
\bibliography{references}

\end{document}
