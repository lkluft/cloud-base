\documentclass{beamer}
\usepackage[ngerman]{babel}
\usepackage[utf8]{inputenc}
\usepackage[T1]{fontenc}
\usepackage{lmodern}

% bibliography
\usepackage{natbib}

% theme and colortheme
\usetheme{Berkeley}
\usecolortheme{spruce}
\setbeamertemplate{caption}[numbered]

% footer setting
\beamertemplatenavigationsymbolsempty
\setbeamertemplate{footline}[frame number]

% math mode settings
\usepackage{mathtools}
\usepackage{amssymb}
\usepackage{euler}
\usepackage[makeroom]{cancel}
\renewcommand{\deg}{\ensuremath{^{\circ}}}
\renewcommand{\epsilon}{\varepsilon}

% Grafikumgebungen
\usepackage{graphicx}
\usepackage{subcaption}
%\captionsetup[subfigure]{labelformat=parens, labelsep=quad}
\usepackage{floatflt}
\usepackage{float}

% table of contents at beginning of each section
%\AtBeginSection[]
%{
%  \begin{frame}
%    \frametitle{Contents}
%    \tableofcontents[currentsection]
%  \end{frame}
%}

% information to be included in the title page
\title{Bestimmung der Wolkenhöhe mittels Pyrgeometer}
\author{Lehrexkursion 2016 - Wolkenfernerkundung}
\date{\today}

%%%%%%%%%%%%%%%%%%%%%%%%%%%%%%%%%%%%%%%%%%%%%%%%%%
\begin{document}
\begin{frame}
\titlepage
\end{frame}

\section{Hintergrund}
\begin{frame}{Hintergrund}
\begin{columns}
\begin{column}{0.6\textwidth}
\begin{itemize}
  \vfill\item Bewölkung erhöht die langwellige Einstrahlung
  \vfill\item Die Strahlungsintensität hängt von der Temperatur des
      emittierenden Körpers ab \[ I \propto T\]
  \vfill\item Strahlungsmessungen enthalten Informationen über die
      Wolkentemperatur und ermöglichen so Rückschlüsse auf die Wolkenhöhe
  \vfill
\end{itemize}
\end{column}

\begin{column}{0.4\textwidth}
\begin{figure}[ht]
    \centering
    \includegraphics[width=1\textwidth]{figures/pyrgeometer.png}
    \caption{Pyrgeometer}
    \label{fig:pyrgeometer}
\end{figure}
\end{column}
\end{columns}
\end{frame}


\section{Pyrgeometer}
\begin{frame}{Pyrgeometer}
\begin{columns}
\begin{column}{0.6\textwidth}
\begin{itemize}
  \vfill\item Messung der atmosphärischen Gegenstrahlung $L\downarrow$\\
              (5 bis 50\,$\mu$m)
  \vfill\item Schwarze Sensoroberfläche mit Abschirmung der kurzwelligen
      Einstrahlung
  \vfill\item Langwellige Nettostrahlung wird durch Wärmeleitung in einer
      Thermosäule ausgeglichen
\vfill
\end{itemize}
\end{column}

\begin{column}{0.4\textwidth}
\begin{figure}[ht]
    \centering
    \includegraphics[width=1\textwidth]{figures/pyrgeometer_funktion.png}
    \caption{Aufbau}
    \label{fig:pyrgeometer_funktion}
\end{figure}
\end{column}
\end{columns}

\begin{alertblock}{Pyrgeometerformel}
  \centering$ L\downarrow = \lambda (T_S - T_G) + \sigma T_G^4 \approx cU + \sigma T_G^4 $
\end{alertblock}
\end{frame}


\section{Konzept}
\begin{frame}{Konzept}
\begin{itemize}
  \vfill\item Berechnung der Wolkentemperatur aus den Strahlungsmessungen des
      Pyrgeometers
  \vfill\item Zuordnung der Wolkentemperatur zu einer Höhe
  \begin{itemize}
    \item adiabatische Abnahme der Temperatur ausgehend\\
          von der Bodentemperatur $T_s$
    \item Standardatmosphäre mit angepasster $T_s$
    \item Radiosondenaufstieg
  \end{itemize}
  \vfill
\end{itemize}
\end{frame}


\section{Strah\-lungs\-trans\-fer}
\begin{frame}{Opazität}
\begin{columns}
\begin{column}{0.45\textwidth}
\begin{itemize}
  \vfill\item Großteil der gemessenen Strahlung entstammt der bodennahen Atmosphäre
  \vfill\item Optisches Fenster zwischen 20-40\,THz erlaubt Blick in höhere
      Atmosphärenschichten
  \vfill
\end{itemize}
\end{column}

\begin{column}{0.55\textwidth}
\begin{figure}[ht]
    \centering
    \includegraphics[width=1\textwidth]{figures/opacity.pdf}
    \caption{Opazität in Abhängigkeit von Frequenz und Höhe.}
    \label{fig:opacity}
\end{figure}
\end{column}
\end{columns}
\end{frame}

\begin{frame}{Radianzspektrum}
\begin{figure}[ht]
    \centering
    \includegraphics[width=0.8\textwidth]{figures/spectrum.pdf}
    \caption{Radianz in Abhängigkeit der Frequenz.}
    \label{fig:spectrum}
\end{figure}
\end{frame}

\begin{frame}{Abschätzung der CLB}
\begin{itemize}
  \vfill\item Differenz der gemessenen LWR und der LWR der bodennahen
      Atmosphäre ist die entscheidende Größe
      \[ \Delta LWR = LWR - \int B_\nu(\nu, T_s) \partial\nu \]
  \vfill\item Die Differenz der Helligkeitstemperaturen gibt anschaulich an,
      wie viel Kelvin das optische Fenster kälter ist als die Temperatur am
      Boden
  \vfill\item Umrechnung in eine Höhe mittels Temperaturgradienten $\gamma$
      \[ CLB_{est} = \frac{\Delta T_{LWR}}{\gamma} \]
  \vfill
\end{itemize}
\end{frame}


\section{Ergebnisse}
\begin{frame}{Ergebnisse}
\begin{itemize}
  \item $CLB_{est}$ im wolkenfreien Fall gibt eine Abschätzung der maximalen
      Detektionshöhe
\end{itemize}
\begin{figure}[ht]
    \centering
    \includegraphics[width=1\textwidth]{figures/ceilometer.png}
    \caption{Zeitreihe der berechneten Wolkenhöhe sowie der Ceilometer-Rückstreuung.}
    \label{fig:clb}
\end{figure}
\end{frame}

\begin{frame}{Fazit}
\begin{itemize}
  \vfill\item Messungen der langwelligen Gegenstrahlung und der bodennahen
      Temperatur (2\,m) ermöglichen eine Abschätzung der Höhe tiefer Bewölkung
  \vfill\item Die maximale Detektionshöhe hängt stark vom Atmosphärenzustand ab
      und liegt zwischen 2300 und 3500\,m
  \vfill\item Variabilität des vertikalen Temperaturgradienten kann die
      Ergebnisse verschlechtern 
\vfill
\end{itemize}
\end{frame}

\begin{frame}{Ausblick}
\begin{itemize}
  \vfill\item Verbesserung des vertikalen Temperaturgradienten über
      Einbeziehung der Bodenfeuchte
  \vfill\item Einschränkung des Pyrgeometer-Blickwinkels (Metallrohr) zur
      besseren Vergleichbarkeit mit Ceilometermessungen
  \vfill\item Wolkenfreie Messungen bieten Informationen über den
      Wasserdampfgehalt der Atmosphäre. Eine Regression von
        Strahlungstransfersimulationen verschiedener Atmosphären bietet die
        Möglichkeit einer Abschätzung der Wasserdampfsäule
  \vfill
\end{itemize}
\end{frame}

\end{document}
